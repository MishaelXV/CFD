\section{Лекция 16 (06.11)}

\subsection{Алгебраический подход к построению нелинейных TVD схем}
TODO

\subsection{Схемы с искусственной вязкостью}
TODO
\subsubsection{Направленная искусственная вязкость}
TODO
\subsubsection{SUPG}
TODO


\subsection{Задание для самостоятельной работы}
\label{sec:hw_supg}
В тестовом примере \cvar{[convdiff-fem-supg]} из файла \ename{convdiff_fem_test.cpp}
производится численного решение одномерного нестационарного уравнения конвекции-диффузии
$$
\dfr{u}{t} + v \dfr{u}{x} - \eps \dfrq{u}{x} = 0
$$
в области $x\in[0, 4]$
с точным решением вида
$$
u^e(x, t) = \frac{1}{\sqrt{4\pi \eps (t + t_0)}} \exp\left(-\frac{(x - v t)^2}{4\eps(t+t_0)}\right)
$$
Точное решение используется для формулировки начальных ($t=0$) и граничных ($x=0,4$) условий первого рода.
Результат расчёта сохраняется в файл \ename{convdiff-supg.vtk.series}.

Задача полудискретизуется по схеме Кранка-Николсон c шагом по времени, вычисленным через число Куранта $\rm C = 0.5$.

После дискретизации задача сводится к СЛАУ относительно неизвестного сеточного вектора $u$
$$
\begin{array}{ll}
\mat A u = \mat B \check u. \\
\mat A = \mat M + \tau\theta \mat K + \eps\tau\theta \mat S, \\
\mat B = \mat M - \tau(1-\theta)\mat K - \eps\tau(1-\theta) \mat S.
\end{array}
$$
Здесь матрица масс $\mat M$ -- результат коненчноэлементной аппроксимации единичного оператора,
матрица переноса $\mat K$ -- конвективного оператора,
а матрица жёсткости $\mat S$ -- оператора диффузии,
$\tau$ -- шаг по времени, и для схемы Кранка--Николсон $\theta=0.5$.


Стабилизированные матрицы вычислялись по следующим формулам:
\begin{equation*}
\begin{WithArrows}
\mat M =& \int\limits_\Omega \phi_j (\phi_i + s \vec v \cdot \nabla \phi_i) \, d\vec x
           \\[10pt]
\mat K =& \arint{\vec v \cdot \nabla \phi_j (\phi_i + s \vec v \cdot \nabla \phi_i)}{\Omega}{\vec x}
           \\[10pt]
\mat S =& -\arint{\nabla^2 \phi_j (\phi_i + s\vec v \cdot \nabla \phi_i)}{\Omega}{\vec x}
           \Arrow{по формуле \cref{eq:partint_laplace_fg}}\\[10pt]
       =& \arint{\nabla \phi_j \cdot \nabla(\phi_i + s\vec v \cdot \nabla \phi_i)}{\Omega}{\vec x}
           \Arrow{если $\phi_i$ линейны, то $\nabla^2\phi_i = 0$}\\[10pt]
       =& \arint{\nabla \phi_j \cdot \nabla\phi_i}{\Omega}{\vec x}
\end{WithArrows}
\end{equation*}
где $s = \mu h / \left | v \right|^2$, $h$ -- характерный линейный размер элемента, а $\mu$ -- параметр SUPG-стабилизации.

Сборки необходимых матриц осуществляется в процедуре \cvar{asseble_solver}.
Локальные матрицы для операторов $\mat M$, $\mat K$ вычисляются с помощью численного интегрирования в процедуре
\cvar{custom_matrix}, которой в качестве аргумента передаётся функция подинтегрального выражения.

Необходимо:
\begin{enumerate}
\item В одномерном тесте \cvar{[convdiff-supg]} с помощью анимированных графиков продемонстрировать наличие осцилляций
      при выбранных параметрах решения при отсутствии стабилизации \cvar{mu_supg = 0}, а так же эффект от SUPG-слагаемого (\cvar{mu_supg > 0});
\item Нарисовать в сравнении результаты расчёта на конечный промежуток времени для различных $\mu$;
\item Подобрать оптимальную величину параметра $\mu$, минимизирующую норму отклонения численного решения от точного;
\item Написать аналогичный тест для двумерного случая (решение в единичном квадрате) c точным решением
$$
u^e(x, t) = \frac{1}{4\pi \eps (t + t_0)} \exp\left(-\frac{(x - v_x t)^2 + (y - v_y t)^2}{4\eps(t+t_0)}\right)
$$
Использовать $\vec v  = (1, 1)$. Взять треугольную сетку, построенную процедурой \ename{trigrid.py}. Проанализировать результаты, полученные на грубой и подробной сетке.
\end{enumerate}

В целом имеющийся решатель не зависит от геометрической размерности задачи.
Для двумерного решателя нужно
\begin{itemize}
\item изменить значение точного решения и скорости (функции \cvar{velocity}, \cvar{nonstat_solution});
\item внести изменения в процедуру постановки граничных условий в функциях \cvar{assemble_solver}, \cvar{assemble_rhs} (сместо двух точек начала и конца одномерной области
необходимо пройтись по всем граничным узлам двумерной сетки);
\end{itemize}
