\subsection{Векторы}

\subsubsection{Обозначение}

Геометрические вектора обозначаются жирным шрифтом $\vec v$.
Скалярные координаты вектора -- через нижний индекс с обозначением
оси координат: $\left(v_x, v_y, v_z\right)$.
Если вектор $\vec u$ -- вектор скорости, то его декартовые координаты
имеют специальное обозначение $\vec u = \left(u, v, w\right)$.
Единичные вектора, соответствующие осям координат, обозначаются 
знаком $\hat\cdot$: $\vec{\hat x}$, $\vec{\hat y}$, $\vec{\hat z}$.
Координатные векторы обозначаются по символу первой оси. Например, $\vec x = (x, y, z)$ или $\vec \xi = (\xi, \eta, \zeta)$.

Операции в векторами имеют следующее обозначение (расписывая в декартовых координатах):
\begin{itemize}
\item
Умножение на скалярную функцию
\begin{equation}
\label{eq:vec_scalar}
f \vec u = (f u_x)\vec{\hat x} + (f u_y)\vec{\hat y} + (f u_z)\vec{\hat z};
\end{equation}
\item
Скалярное произведение
\begin{equation}
\label{eq:vec_dot}
\vec u \cdot \vec v = u_x v_x + u_y v_y + u_z v_z;
\end{equation}
\item
Векторное произведение
\begin{equation}
\label{eq:vec_cross}
\vec u\times\vec v = 
\left|
\begin{array}{ccc}
\vec{\hat x} & \vec{\hat y} & \vec{\hat z} \\
u_x & u_y & u_z \\
v_x & v_y & v_z
\end{array}
\right| = 
\left(u_y v_z - u_z v_y\right)\vec{\hat x} -
\left(u_x v_z - u_z v_x\right)\vec{\hat y} +
\left(u_x v_y - u_y v_x\right)\vec{\hat z}.
\end{equation}

\end{itemize}

В двумерном случае можно считать, что $u_z = v_z = 0$.
Тогда результатом векторного произведения согласно \cref{eq:vec_cross} 
будет вектор, направленный перпендикулярно плоскости $xy$:
$$
\vec u \times \vec v = (u_x v_y - u_y v_x)\vec{\hat z}.
$$
При работе с двумерными задачами, где ось $\vec z$ отсутствует,
обычно результатом векторного произведения считают скаляр
\begin{equation}
\label{eq:vec_cross_2d}
2D: \; \vec u \times \vec v = u_x v_y - u_y v_x.
\end{equation}
Геометрический смысл этого скаляра: площадь
параллелограмма, построенного на векторах $\vec u$ и $\vec v$.


\subsubsection{Набла--нотация}

Символ $\nabla$ -- есть псевдовектор, который выражает
покоординатные производные.
Для декартовой системы координат $(x, y, z)$ он запишется в виде
$$
\nabla = \left( \dfr{}{x}, \; \dfr{}{y}, \; \dfr{}{z} \right).
$$
В радиальной $(r, \phi, z)$:
$$
\nabla = \left( \dfr{}{r}, \; \frac{1}{r}\dfr{}{\phi}, \; \dfr{}{z} \right).
$$
В цилиндрической $(r, \theta, \phi)$:
$$
\nabla = \left( \dfr{}{r}, \; \frac{1}{r}\dfr{}{\theta}, \; \frac{1}{r\sin\theta}\dfr{}{\phi} \right).
$$
Удобство записи дифференциальных выражений с использованием $\nabla$ заключается в независимости записи от
вида системы координат.
Но если требуется обозначить производную по конкретной координате,
то, по аналогии с обычными векторами, это делается через нижний индекс:
$$
\nabla_n f = \dfr{f}{n}.
$$

Для этого символа справедливы все векторные операции, описанные ранее.
Так, применение $\nabla$ к скалярной функции аналогично умножению вектора
на скаляр \cref{eq:vec_scalar} (здесь и далее приводятся покоординатные выражения для декартовой системы):
\begin{equation}
\label{eq:del_grad}
\nabla f  = \left(\nabla_x f, \; \nabla_y f, \; \nabla_z f\right) = \dfr{f}{x}\vec{\hat x} + \dfr{f}{y}\vec{\hat y} + \dfr{f}{z}\vec{\hat z}.
\end{equation}
Результатом этой операции является вектор.

Скалярное умножение $\nabla$ на вектор $\vec v$ по аналогии с \cref{eq:vec_dot} -- есть дивергенция:
\begin{equation}
\label{eq:del_div}
\nabla \cdot \vec v  = \dfr{v_x}{x} + \dfr{v_y}{y} + \dfr{v_z}{z}
\end{equation}
результат которой -- скалярная функция.

Двойное применение $\nabla$ к скалярной функции -- это оператор Лапласа:
\begin{equation}
\label{eq:del_laplace}
\nabla \cdot \nabla f  = \nabla^2 f = \dfrq{f}{x} + \dfrq{f}{y} + \dfrq{f}{z}
\end{equation}

Ротор -- аналог векторного умножнения \cref{eq:vec_cross}:
\begin{equation}
\label{eq:del_rotor}
\nabla\times\vec v = 
\left|
\begin{array}{ccc}
\vec{\hat x} & \vec{\hat y} & \vec{\hat z} \\
\nabla_x & \nabla_y & \nabla_z \\
v_x & v_y & v_z
\end{array}
\right| = 
\left(\dfr{v_z}{y} - \dfr{v_y}{z}\right)\vec{\hat x} -
\left(\dfr{v_z}{x} - \dfr{v_x}{z}\right)\vec{\hat y} +
\left(\dfr{v_y}{x} - \dfr{v_x}{y}\right)\vec{\hat z}.
\end{equation}
