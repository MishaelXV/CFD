\section{Лекция 19 (7.12)}

\subsection{Схема PISO}
\subsubsection{Итерации PISO}
\label{sec:piso}
Рассмотрим модификацию схемы SIMPLE (п.~\ref{sec:simple-algo}), использующую несколько поправок скорости и давления.
В качестве исходных уравнений используем полудискретизованную схему \cref{eq:ns2d_semi_u,eq:ns2d_semi_v,eq:ns2d_semi_div}.
По аналогии с \cref{eq:ns2d_decomp} запишем искомое решение в виде пробной функции и корректоров
\begin{equation}
\label{eq:piso_decomp}
\begin{array}{l}
    \hat u = u^* + u' + u'' + u''' + \dots,\\
    \hat v = v^* + v' + v'' + v''' + \dots,\\
    \hat p = p + p' + p'' + p''' + \dots.
\end{array}
\end{equation}
Дополнительно положим, что все поправки скорости выше первой бездивергентны:
\begin{equation}
\label{eq:piso_div_uprime2}
\nabla\cdot \vec u'' = 0, \quad \nabla\cdot \vec u''' = 0, \quad \dots
\end{equation}
Тогда из требования бездивирегности финальной скорости $\hat u$ получим 
\begin{equation}
\label{eq:piso_div_uprime1}
\nabla \cdot (\vec u^* + \vec u' + \vec u'' + \dots ) = 0 \hence \nabla \cdot \vec u' = - \nabla \cdot \vec u^*
\end{equation}

Для удобство записи введём обозначения для линеаризованных операторов, действующих на функции скорости
\begin{equation}
\label{eq:piso_s}
\begin{aligned}
&S^u(\hat u) = \hat u + \tau\dfr{u \hat u}{x} + \tau\dfr{v \hat u}{y} - \frac{\tau}{\Ren}\left(\dfrq{\hat u}{x} + \dfrq{\hat u}{y}\right),\\[5pt]
&S^v(\hat v) = \hat v + \tau\dfr{u \hat v}{x} + \tau\dfr{v \hat v}{y} - \frac{\tau}{\Ren}\left(\dfrq{\hat v}{x} + \dfrq{\hat v}{y}\right).\\[5pt]
\end{aligned}
\end{equation}
Разложим эти операторы как
\begin{equation*}
S^u(u) = A^u(u) + H^u(u),
\end{equation*}
где $A^u$ -- оператор, который в дискретизованной форме имеет диагоноальную структуру, а $H^u$ -- оператор, имеющий в дискретизованном виде нулевую диагональ.

Далее подставим разложения \cref{eq:piso_decomp} в определяющие уравнение \cref{eq:ns2d_semi_u,eq:ns2d_semi_v}.
Декомпозицию полученного уравнения проведём следующим образом (на примере уравнения для $u$):
\begin{alignat}{6}
& & A^u(u^*)    & + H^u(u^*)     &  =  & -\tau\ddfr{p}{x}    & + u     \label{eq:piso_eq_ustar}\\[5pt]
&+& A^u(u')     &                &  =  & -\tau\ddfr{p'}{x}   &         \label{eq:piso_eq_uprime1}\\[5pt]
&+& A^u(u'')    & + H^u(u')      &  =  & -\tau\ddfr{p''}{x}  &         \label{eq:piso_eq_uprime2}\\[5pt]
&+& A^u(u''')   & + H^u(u'')     &  =  & -\tau\ddfr{p'''}{x} &         \label{eq:piso_eq_uprime3}\\[5pt]
& &             &                &\dots&                     &\nonumber\\[10pt]
&=& A^u(\hat u) & + H^u(\hat u)  &  =  & -\tau\ddfr{\hat p}{x} & + u \nonumber
\end{alignat}
Уравнение \cref{eq:piso_eq_ustar} аналогично уравнению \cref{eq:ns2d_ustar} метода SIMPLE.
Оно используется для определения пробной скорости $u^*$.
Уравнение для первой поправки скорости \cref{eq:piso_eq_uprime1} (аналог \cref{eq:ns2d_uprime})
совместно с уравнением неразрывности \cref{eq:piso_div_uprime1} используется для формулировки задачи Пуассона для $p'$ \cref{eq:ns2d_pprime_diff}:
\begin{equation}
\label{eq:piso_eq_p_prime1}
- \nabla \cdot \tensor d \nabla p' = - \frac1\tau \nabla \cdot \vec u^*.
\end{equation}
Здесь диагональный тензор $\tensor d$ равен
\begin{equation*}
\tensor d = \left(
\begin{array}{cc}
(A^u)^{\sminus 1} & 0                \\
0                 & (A^v)^{\sminus 1}
\end{array}
\right)
\end{equation*}
Уравнения \cref{eq:piso_eq_uprime2,eq:piso_eq_uprime3} и последующие используются для формулировки задачи для $p'$, $p''$ и т.д.
Рассмотрим его получение на примере \cref{eq:piso_eq_uprime2}.
Во-первых, воспользуемя диагональностью (то есть легкообратимостью) оператора $A^u$ и выразим $u''$:
\begin{equation*}
u'' = -\tau (A^u)^{\sminus 1}\dfr{p''}{x} - (A^u)^{\sminus 1} H^u u'.
\end{equation*}
Аналогичное уравнение можно вывести и для вертикальной компоненты скорости:
\begin{equation*}
v'' = -\tau (A^v)^{\sminus 1}\dfr{p''}{y} - (A^v)^{\sminus 1} H^v v'.
\end{equation*}
Отсюда можно записать градиент от $\vec u''$:
\begin{equation*}
\nabla \cdot \vec u'' = -\tau \nabla \cdot \tensor d \nabla p'' - \nabla \cdot \tensor d \, \tensor H \vec u',
\qquad
\tensor H = \left(\begin{array}{cc}
H^u & 0 \\
0   & H^v
\end{array}
\right).
\end{equation*}
Этот градиент согласно \cref{eq:piso_div_uprime2} равен нулю.
Похожих рассуждений будем придерживаться для вывода уравнений для $p'''$ и далее.
Итого, уравнения для второй и последующих поправок далвения давления примут вид
\begin{align}
\label{eq:piso_eq_p_prime2}
 - &\nabla \cdot \tensor d \nabla p'' = \frac1\tau \nabla \cdot \tensor d \, \tensor H \vec u', \\
 - &\nabla \cdot \tensor d \nabla p''' = \frac1\tau \nabla \cdot \tensor d \, \tensor H \vec u'', \nonumber\\
   &\dots \nonumber
\end{align}

Последовательность шагов при решении задачи на временном слое примет вид
\begin{enumerate}
\item Даны значение $\vec u$, $\vec v$, $p$ с прошлого слоя,
\item Собираем сеточные операторы $\tensor A$, $\tensor H$ как диагональную и внедиагональную часть
      линеаризованного оператора $\tensor S$ \cref{eq:piso_s}, и обратный к $\tensor A$ оператор $\tensor d$,
\item Из уравнения \cref{eq:piso_eq_ustar} и его аналога для вертикальной компоненты скорости находим $u^*$, $v^*$,
\item Из уравнения \cref{eq:piso_eq_p_prime1} находим первую поправку для давления $p'$,
\item Из уравнения \cref{eq:piso_eq_uprime1}  и его аналога для вертикальной компоненты скорости находим $u'$, $v'$,
\item Из уравнения \cref{eq:piso_eq_p_prime2} находим второую поправку для давления $p''$,
\item Из уравнения \cref{eq:piso_eq_uprime2}  и его аналога для вертикальной компоненты скорости находим $u''$, $v''$,
\item Повторяем шаги 6, 7 для каждой поправки скорости и давления выше первой,
\item Записываем окончательное решение  на слое используя \cref{eq:piso_decomp}.
\end{enumerate}

Первые 5 шагов представленного алгоритма полностью соответствуют ранее рассмотренному методу SIMPLE.
Шаги 6, 7 -- есть дополнительные уточняющие PISO итерации.
В отличии от SIMPLE-итераций, эти итерации не требуют пересборки операторов и дополнительной инициализации решателей систем
линейных уравнений, поэтому они более эффективны с точки зрения скорости расчёта.

Отметим, что в представленной постановке (в отличии от схемы SIMPLE из п.~\ref{sec:simple-algo}) мы не использовали релаксацию, поэтому эту схему можно использовать
для решения нестационарный задачи \cref{eq:ns2d_nonstat}, трактуя $\tau$ как шаг по реальному времени.

\subsubsection{Граничные условия}
TODO

\subsubsection{Схема с внешними итарциями PIMPLE}
TODO

\subsection{Конвективный теплообмен}
\subsubsection{Уравнение теплопроводности}

Дополним нестационарную систему уравнений течения \cref{eq:ns2d_nonstat} уравнением теплообмена
\begin{equation*}
\rho c_p\left(\dfr{T}{t} + u\dfr{T}{x} + v\dfr{T}{y}\right) = \lambda\left(\dfrq{T}{x} + \dfrq{T}{y}\right).
\end{equation*}
Здесь $T$ -- температура течения, К; $\rho$ -- плотность жидкости, кг/м$^3$; $c_p$ -- теплоёмкость, Дж/кг/К;
$\lambda$ -- теплопроводность, Вт/м/К.

В безразмерном виде это уравнение примет вид
\begin{equation}
\label{eq:ns2d_nonstat_temperature}
\dfr{T}{t} + u\dfr{T}{x} + v\dfr{T}{y} = \frac{1}{\Pen}\left(\dfrq{T}{x} + \dfrq{T}{y}\right).
\end{equation}
где безразмерная температура $T$ вычислена через размерную $T^{dim}$ как 
$$
T = \frac{T^{dim} - T^0}{\triangle T},
$$
а число Пекле $\Pen$ есть
$$
\Pen = \frac{U L}{a}, \quad a = \frac{\lambda}{\rho c_p}.
$$

\subsubsection{Дискретизация по времени}

Пользуясь обозначениями из п. \ref{sec:simple-nonstat-algo}
запишем неявную дискретизацию по времени уравнения \cref{eq:ns2d_nonstat_temperature} в виде

\begin{equation}
\label{eq:ns2d_nonstat_temperature_semi}
\frac{\hat T - \check T}{\dt} + \hat u\dfr{\hat T}{x} + \hat v\dfr{\hat T}{y} = \frac{1}{\Pen}\left(\dfrq{\hat T}{x} + \dfrq{\hat T}{y}\right).
\end{equation}

Полученное уравнение не содержит значений $u, v$ с текущего итерационного слоя,
и значит может быть решено один раз в конце шага по времени, когда сходимость уже достигнута.

\subsubsection{Аппроксимация на разнесённой сетке}

Пространственную аппроксимацию уравнения \cref{eq:ns2d_nonstat_temperature_semi}
будем проводить на разнесённой сетке в центральных (``чёрных'') узлах сетки (\figref{fig:staggered_grid}).
При этом конвективную производную будем приближать с помощью симметричной разности.
Полученная конечная разность для узла $i+\tfrac12, j+\tfrac12$ примет вид
\begin{align}
\label{eq:ns2d_nonstat_temperature_scheme}
\frac{1}{\dt}\hat T_{i+\tfrac12,j+\tfrac12}
    & + \hat u_{i+\tfrac12, j+\tfrac12}\frac{\hat T_{i+\tfrac32, j+\tfrac12} - \hat T_{i-\tfrac12, j+\tfrac12}}{2h_x}\\
    \nonumber
    & + \hat v_{i+\tfrac12, j+\tfrac12}\frac{\hat T_{i+\tfrac12, j+\tfrac32} - \hat T_{i+\tfrac12, j-\tfrac12}}{2h_y}\\
    \nonumber
    & + \frac{1}{\Pen} \frac{-\hat T_{i+\tfrac32, j+\tfrac12} + 2\hat T_{i+\tfrac12,j+\tfrac12} - \hat T_{i-\tfrac12, j+\tfrac12}}{h_x^2} \\
    \nonumber
    & + \frac{1}{\Pen} \frac{-\hat T_{i+\tfrac12, j+\tfrac32} + 2\hat T_{i+\tfrac12,j+\tfrac12} - \hat T_{i+\tfrac12, j=\tfrac12}}{h_x^2} \\
    \nonumber
    & = \frac{1}{\dt}\check T_{i+\tfrac12,j+\tfrac12}.
\end{align}

Значения компонент скорости в центрах ячеек вычисляются с помощью ближайщей полусуммы
\begin{align*}
&\hat u_{i+\tfrac12, j+\tfrac12} \approx \frac{\hat u_{i, j+\tfrac12} + \hat u_{i+1, j+\tfrac12}}{2}, \\
&\hat v_{i+\tfrac12, j+\tfrac12} \approx \frac{\hat v_{i+\tfrac12, j} + \hat v_{i+\tfrac12, j+1}}{2}.
\end{align*}


\subsubsection{Граничные условия}

Учёт граничных условий производится за счёт вычисления значений в фиктивных узлах около границ.

Пусть требуется учесть условие на левой стенке ($i=0$).
Тогда соответствующий фиктивный узел будет иметь индекс $-\tfrac12, j$.
Ниже приведём его выражения для трёх типов граничных условий.

\subsubsubsection{Условия первого рода}
Пусть
\begin{equation}
\label{eq:ns2d_temperature_bc1}
\left. T\right|_{left} = T^{\Gamma}
\end{equation}
Тогда
$$
\frac{T_{-\tfrac12, j+\tfrac12} + T_{\tfrac12, j+\tfrac12}}{2} = T^{\Gamma}.
$$
Отсюда 
$$
T_{-\tfrac12, j+\tfrac12} = -T_{\tfrac12, j+\tfrac12} + 2 T^{\Gamma}.
$$
Таким образом, если в матрицу $A^T$ левой части выражения \cref{eq:ns2d_nonstat_temperature_scheme}
в фиктивную колонку $k\left[-\tfrac12, j+\tfrac12\right]$ требуется добавить
какое-то значение $a$, это равносильно добавлению этого выражнеия с обратным знаком в диагональ
и удвоенного выражения, умноженного на граничное значение, в правую часть $b^T$:
\begin{align}
\nonumber
&k_0 = k\left[\tfrac12, j + \tfrac12\right], \quad k_1 = k\left[-\tfrac12, j + \tfrac12\right], \\
\label{eq:ns2d_temperature_bc1_scheme}
&A^T_{k_0, k_1} {{+}{=}} a \hence
     A^T_{k_0, k_0} \minuseq a, \quad b^T_{k_0} \minuseq 2 a T^\Gamma.
\end{align}


\subsubsubsection{Условия второго рода}
Если на левой границе задано условие второго рода
\begin{equation}
\label{eq:ns2d_temperature_bc2}
\left.\dfr{T}{n}\right|_{left} = -\left. \dfr{T}{x} \right|_{left} = q
\end{equation}
То вычисление фиктивного узла производится из конечной разности вида
$$
\frac{T_{-\tfrac12, j+\tfrac12} - T_{\tfrac12, j+\tfrac12}}{h_x} = q.
$$
Отсюда 
$$
T_{-\tfrac12, j+\tfrac12} = T_{\tfrac12, j+\tfrac12} + h_x q
$$
Тогда
\begin{align}
\label{eq:ns2d_temperature_bc2_scheme}
&A^T_{k_0, k_1} {{+}{=}} a \hence
     A^T_{k_0, k_0} \pluseq a, \quad b^T_{k_0} \minuseq h_x q
\end{align}

\subsubsubsection{Условия третьего рода}
Пусть на левой границе задано условие второго рода
\begin{equation}
\label{eq:ns2d_temperature_bc3}
\left.\dfr{T}{n}\right|_{left} = -\left. \dfr{T}{x} \right|_{left} = \alpha T + \beta
\end{equation}

Расписывая производную и вычисляя значение температуры на стенке через полусумму, получим
$$
\frac{T_{-\tfrac12, j+\tfrac12} - T_{\tfrac12, j+\tfrac12}}{h_x} = \alpha \frac{T_{-\tfrac12, j+\tfrac12} + T_{\tfrac12, j+\tfrac12}}{2} + \beta.
$$
Отсюда выразим значение в фиктивном узле
$$
T_{-\tfrac12, j+\tfrac12} = \frac{2 + \alpha h_x}{2-\alpha h_x} T_{\tfrac12, j+\tfrac12} + \frac{2\beta h_x}{2 - \alpha h_x}
$$
Тогда
\begin{align}
\label{eq:ns2d_temperature_bc3_scheme}
&A^T_{k_0, k_1} {{+}{=}} a \hence
     A^T_{k_0, k_0} \pluseq \frac{2 + \alpha h_x}{2 - \alpha h_x} a, \quad b^T_{k_0} \minuseq \frac{2 \beta h_x}{2 - \alpha h_x} a.
\end{align}

\subsubsubsection{Универсальность условий третьего рода}
Условие третьего рода \cref{eq:ns2d_temperature_bc3} можно использовать для моделирования условий первого и второго рода.
Так, условия второго рода \cref{eq:ns2d_temperature_bc2} получаются, если положить $\alpha = 0$, $\beta = q$.
А условия первого \cref{eq:ns2d_temperature_bc1}, -- если $\alpha = \eps^{-1}, \beta = -\eps^{-1}T^\Gamma$, где $\eps$ -- малое положительное число.

Если подставить эти выражения в формулу \cref{eq:ns2d_temperature_bc3_scheme}, то можно убедится,
что они дадут выражения \cref{eq:ns2d_temperature_bc2_scheme} и \cref{eq:ns2d_temperature_bc1_scheme} (в пределе при $\eps \to 0$)
соответственно.

\subsubsection{Коэффициент теплообмена}
На границах, где заданы условия первого рода \cref{eq:ns2d_temperature_bc1}
можно вычислить тепловой поток, тем самым
определив, сколько тепловой энергии требуется для
поддержания этой постоянной температуры.

Безразмерный интегральный коэффициент теплообмена (интегральное число Нуссельта) определяется как
\begin{equation}
\label{eq:integral_nu}
\Nun = \arint{\dfr{T}{n}}{\gamma}{s}.
\end{equation}

Для получения размерной мощности из этого безразмерного коэффициента (измеряемой в Ваттах),
необходимо умножить его на $\lambda \triangle T L$.

Вычисление интегрального числа Нуссельта из определения \cref{eq:integral_nu} происходит по той
же схеме, что и вычисление коэффициентов сил \cref{eq:ns2d_gamma_quadrature}.
При этом нормальная производная на границе $\dsfr{T}{n}$ вычисляется в виде
\begin{equation}
\label{eq:dtdn_scheme}
(x,y)\in\gamma_i: \quad \dfr{T}{n} \approx \frac{T^\Gamma - T_k}{h/2},
\end{equation}
где $\gamma_i$ -- отрезок границы, $k$ -- индекс ячейки, прилегающей к этому отрезку,
$h$ -- шаг сетки, поперёк границы ($h_x$ для вертикальных границ и $h_y$ -- для горизонтальных).


\subsection{Задание для самостоятельной работы}
\clisting{open}{"test/heat_transfer_piso_test.cpp"}
В тесте \cvar{[heat-transfer-piso]} файла \ename{heat_transfer_piso_test.cpp} представлено решение нестационарной задачи обтекания
нагреваемого тела, расчитанного по схеме PISO (п.~\ref{sec:piso}).
На основе этого теста нужно решить задачу обтекания двух тел: одно расположено
в прямоугольнике
$$
\gamma_1: \begin{cases}
2.0 \leq x \leq 2.5, \\
-0.7 \leq y \leq 0.3,
\end{cases}
$$
второе --
$$
\gamma_2: \begin{cases}
4 \leq x \leq 4.5, \\
-0.3 \leq y \leq 0.7.
\end{cases}
$$
Использовать условия для температуры:
\begin{align*}
(x, y) \in \Gamma_{in}: \quad &T= 0, \\
(x, y) \in \gamma_1: \quad &T=0.5,\\
(x, y) \in \gamma_2: \quad &T=1.0.
\end{align*}
Область расчёта $[0,15]\times[-2,2]$:
\begin{cppcode}
RegularGrid2D grid(0, 15, -2, 2, 15*n_unit, 4*n_unit);
\end{cppcode}
Остальные граничные условия использовать те же, что и
в рассмотренной в п. \ref{sec:prob-obstacle-temp} задаче.
Параметры задачи:
$\Ren = 100$, $\Pen = 100$, $\dt = 0.05$, $t_{end} = 200$, $n=10$.

Необходимо:
\begin{itemize}
\item Нарисовать анимированную картину течения: скорость, давление и температуру
\item Представить графики изменения коэффициентов сопротивления и теплообмена во времени для обоих обтекаемых тел
\item Увеличить количество PISO итераций (например, до \cvar{n_piso=3}) и сравнить полученные коэффицинеты сопротивления с случаем одной итерации.
\end{itemize}

\paragraph{Задание сетки с неактивными ячейками}
\clisting{to-start}{}
\clisting{pass}{"TEST_CASE"}
Обтекаемые препятствия следует задавать
при определении сетки. В рассмотренном примере из предыдущего пункта
это делалось в строке
\clisting{line}{"deactivate_cells"}
В настоящей задачи нужно эту функцию вызвать два раза, указав там по очереди обе нужные области.

\paragraph{Задание граничных условий на температуру}
Поскольку в задаче граничные условия на обтекаемых
телах отличаются по своему значению, то следует
модифицировать алгоритм их задания.
Граничные условия на температуру задаются в функции \cvar{compute_temperature}
в строке
\clisting{to-start}{}
\clisting{pass}{"HeatTransferPisoWorker::compute_temperature()"}
\clisting{lines-range}{"internal boundary", "rhs"}

Согласно форме
\cref{eq:ns2d_temperature_bc1_scheme}
изменения в левой части не зависит от величины граничной температуры,
а в провой -- пропорционально ей.
Таким образом, для первого тела (где $T^\Gamma = 0.5$) добавка в правую часть будет иметь вид
\begin{minted}[linenos=false]{c++}
rhs[row_index] -= 2*value*0.5,
\end{minted}
а для второго -- останется такой же, как и раньше.

При этом нужно уметь отличать грани, принадлежащие
первому телу от граней, принадлежащих второму.
Для этого в классе \cvar{HeatTransferPisoWorker} можно
объявить функцию, которая определяет ближайшую к ячейке границу.
Саму функцию можно реализовать просто используя координаты
центра ячейки \cvar{_grid.cell_center(icell)}. Например:
\begin{minted}[linenos=false]{c++}
// => 1 если ячейка icell близка к первому обтекаемому телу и 2 - если ко второму
int HeatTransferPisoWorker::gamma_closest_to_cell(size_t icell){
	double x = _grid.cell_center(icell).x();
	if (x < 3.25) {  // центр между первым и вторым телами
		return 1;
	} else {
		return 2;
	}
}
\end{minted}
Тогда можно написать
\begin{minted}[linenos=false]{c++}
double t_gamma = (gamma_closest_to_cell(row_index) == 1) ? 0.5 : 1.0;
rhs[row_index] -= 2*t_gamme*value;
\end{minted}
Здесь запись \cvar{double a = (cond) ? 0.5 : 1.0;} есть сокращение от

\begin{minted}[linenos=false]{c++}
double a;
if (cond){
	a = 0.5;
} else {
	a = 1.0
}
\end{minted}

\paragraph{Вычисление коэффициентов}
Во-первых следует модифицировать структуру, хранящую коэффициенты,
сделав там отдельные записи для каждого тела:
\begin{minted}[linenos=false]{c++}
struct Coefficients{
	double cpx1, cpx2;
	double cfx1, cfx2;
	double cx1, cx2;
	double nu1, nu2;
};
\end{minted}
Во-вторых, в функции сохранения этих коэффициентов в файл
в функции \cvar{HeatTransferPisoWorker::save_current_fields}
\begin{minted}[linenos=false]{c++}
cx_writer << time << " ";
cx_writer << coefs.cx1 << " " << coefs.nu1 << " ";
cx_writer << coefs.cx2 << " " << coefs.nu2 << std::endl;
\end{minted}
Соответственно можно поправить легенду в функции \cvar{initialize_saver()}.

Сами коэффициенты следует вычислять в функции \cvar{coefficients()}.
Там нужно завести аггрегаторы на оба тела:
\begin{minted}[linenos=false]{c++}
double sum_cpx1 = 0;
double sum_cfx1 = 0;
double sum_nu1  = 0;
double sum_cpx2 = 0;
double sum_cfx2 = 0;
double sum_nu2  = 0;
\end{minted}

И далее заполнять их в зависимости от близости ячейки.
Так же следует учесть значение граничной температуры
при вычислении $\dsfr{T}{n}$ по формуле \cref{eq:dtdn_scheme}

Например, для вертикальных граней
после определения ячейки \cvar{left_cell}:
\begin{minted}[linenos=false]{c++}
int gamma_i = gamma_closest_to_cell(left_cell);
double t_gamma = (gamma_i == 1) ? 0.5 : 1.0;
\end{minted}
далее учесть при вычислении \cvar{dtdn} (два раза)
\begin{minted}[linenos=false]{c++}
dtdn = (t_gamma - _t[left_cell]) / (_hx/2.0);
\end{minted}
а также при выборе агрегатора
\begin{minted}[linenos=false]{c++}
if (gamma_i == 1){
	sum_cpx1 += pnx * _hy;
	sum_nu1 += dtdn * _hy;
} else {
	sum_cpx2 += pnx * _hy;
	sum_nu2 += dtdn * _hy;
}
\end{minted}

Аналогичную процедуру следует проделать и для горизонтальных граней.

В конце нужно правильным образом заполнить все поля
переменной \cvar{coef}.
\begin{minted}[linenos=false]{c++}
	coefs.cpx1 = 2.0*sum_cpx1;
	coefs.cpx2 = 2.0*sum_cpx2;
	...
\end{minted}
