\section{Лекция 13 (11.05)}
\subsection{Вычисление элементных интегралов в параметрическом пространстве}
Будем вычислять элементные интегралы \cref{eq:fem_mass_matrix,eq:fem_stiff_matrix,eq:fem_load_vector}
в параметрическом пространстве $\vec \xi$.
Для этого введём преобразование координат $\vec x \rightarrow \vec \xi$ согласно п.\ref{sec:coo_transform}.
Интеграл для определения локальной матрицы масс \cref{eq:fem_mass_matrix} 
в параметрическом пространстве распишется согласно формуле \cref{eq:dxideta_integral}
\begin{equation}
\label{eq:mass_matrix}
m^k_{ij} = \arint{\tilde \phi^{(k)}_i(\vec \xi) \tilde \phi^{(k)}_j(\vec \xi)|J^{(k)}(\vec \xi)|}{\tilde E^k}{\vec \xi}
\end{equation}
Здесь $\tilde E^k$ -- параметрический образ конечного элемента $E^k$,
$J^{(k)}$ - Якобиан преобразования для $k$-ого элемента,
$\tilde \phi_{i}^{(k)}$ -- часть базисной функции $\phi_{c^k_i}$, определённая
в конечном элементе $E^k$ и заданная в параметрических координатах,
а $c^k_i$ -- глобальный индекс базисной функции, которая в $k$-ом элементе имеет локальный индекс $i$.
Таким образом справедливо
\begin{equation}
\nonumber
\tilde \phi_{i}^{(k)}(\vec \xi) = \phi_{c^k_i}(\vec x(\vec \xi))
\end{equation}
Локальный вектор нагрузок
\begin{equation}
\label{eq:load_vector}
l^k_{i} = \arint{\tilde \phi^{(k)}_i(\vec \xi)|J^{(k)}(\vec \xi)|}{\tilde E^k}{\vec \xi}
\end{equation}
Локальная матрица жёсткости
\begin{equation}
\label{eq:stiff_matrix}
s^k_{ij} = \arint{\nabla_{\vec x} \tilde \phi^{(k)}_i \cdot \nabla_{\vec x} \tilde \phi^{(k)}_j |J^{(k)}(\vec \xi)|}{\tilde E^k}{\vec \xi}
\end{equation}
Здесь $\nabla_{\vec x} \tilde \phi^{k}_i$ -- градиент локального базиса (заданного в параметрическом пространстве) по физическим координатам.
Для его вычисления следует воспользоваться формулами \cref{eq:vec_grad_dx}
\subsection{Двумерное уравнение Пуассона}
\subsubsubsection{Треугольный элемент. Линейный двумерный базис}
Матрица масс (из \cref{eq:mass_matrix}):
\begin{equation}
\label{eq:mass_matrix_lintri}
M^E_{ij} = \int\limits_0^1 \int\limits_0^{1-\xi} \phi_i(\xi, \eta) \phi_j(\xi, \eta) |J| \, d\eta d\xi =
\frac{|J|}{24}\left(
\begin{array}{ccc}
2 & 1 & 1 \\
1 & 2 & 1 \\
1 & 1 & 2
\end{array}
\right)
\end{equation}

\subsection{Разбор программной реализации МКЭ}
\label{sec:fem_programming}
\clisting{open}{"test/poisson_fem_solve_test.cpp"}
Численное решение уравнения Пуассона с граничными
условиями первого рода реализовано в файле
\ename{poisson_fem_solve_test.cpp}.
Будем рассматривать решение одномерной задачи с использованием пирамидальных базисов (тест \ename{[poisson1-fem-lintri]}).
В этом тесте определяется двумерная аналитическая функция
$$
f(x) = \sin(10 x^2),
$$
и формулируется уравнение Пуассона с граничными условиями первого рода, для которого эта функция является точным решением.
Далее уравнение Пуассона решается численно и полученный численный результат сравнивается
с точным ответом. Норма полученной ошибки печатается в консоль.

В функции верхнего уровня происходит построение одномерной
сетки, создание рабочего объекта, вызов решения с возвращением нормы полученной ошибки
и вывод данныех (сохранение решения в vtk-файл и печать нормы в консоль):
\clisting{pass}{"[poisson1-fem-linsegm]"}
\clisting{lines-range}{"Grid1D", "std::cout"}
Основная работа происходит в классе \cvar{TestPoissonLinearSegmentWorker}.

\subsubsection{Рабочий объект}
Класс \cvar{TestPoissonLinearSegmentWorker}
наследуется от \cvar{ITestPoisson1FemWorker}. В этом классе
сформулированы аналитические функции, служащие
правой частью, точным решением и условиями первого рода уравнения Пуассона.
А этот класс в свою очередь наследуется от \cvar{ITestPoissonFemWorker},
в котором и происходит решение уравнения.
\clisting{to-start}{}
\clisting{block}{"double ITestPoissonFemWorker::solve()"}
Для получения решения сначала собирается левая и правая
часть системы линеных уравнений,
потом происходит учёт граничных условий первого рода
, вызывается решатель системы уравнений и вычислитель нормы ошибки.

Функция сборки матрицы левой части реализует сборку глобальной матрицы жёсткости
через набор локальных матриц
\clisting{to-start}{}
\clisting{block}{"CsrMatrix ITestPoissonFemWorker::approximate_lhs() const"}
Основой для сборки служит специальный объёкт \cvar{_fem} класса
\cvar{FemAssembler} -- сборщик.
Этот объёкт сначала используется для задания шаблона итоговой матрицы,
потом в цикле по элементам вычисляются локальные матрицы и с
помощью метода этого класса \cvar{FemAssembler::add_to_global_matrix}
локальные матрицы добавляются в глобальную.

По аналогичной процедуре работает и сборка правой части \cvar{approximate_rhs}.

\subsubsection{Конечноэлементный сборщик}
Конечноэлементный сборщик \cvar{FemAssembler} -- основной класс, хранящий
всю информацию о текущей конечноэлементной аппроксимации: 
массив конечных элементов и их связность.
Эта информация подаётся ему при конструировании (реализация в файле \ename{cfd/fem/fem_assembler.hpp}).
\clisting{open}{"cfd/fem/fem_assembler.hpp"}
\clisting{lines-range}{"FemAssembler(", ");"}
Связность \cvar{tab_elem_basis} имеет формат \quo{элемент-глобальный базис} и
определяет глобальный индекс для каждого локального базисного индекса.
В рассматренных нами узловых конечных элементах базис связан с узлом сетки.
То есть эта таблица -- это связность локальной и глобальной нумерации узлов сетки для каждой ячейки сетки.

Конечноэлементный сборщик создаётся в методе
\cvar{TestPoissonLinearSegmentWorker::build_fem} итогового
рабочего класса (то есть сборщик специфичен для конкретной сетки
и конкретного выбора типов элементов). Далее он пробрасывается в конструктор базового рабочего класса.

\subsubsection{Концепция конечного элемента}
\clisting{open}{"cfd/fem/fem_element.hpp"}
Класс конечного элемента \cvar{FemElement} определён в файле
\ename{fem/fem_element.hpp} как
\clisting{block}{"struct FemElement"}
Главная задача объекта этого класса -- вычисление элементных матриц,
которые впоследствии используются сборщиком
для создания глобальных матриц.
Для расчёта элементных матриц в свою очередь требуется
\begin{itemize}
\item Геометрия элемента, включающая в себя правило отображения элемента из физической в параметрическую область,
\item Набор локальных базисных функций, заданных в параметрическом пространстве на указанной геометрии,
\item Непосредственно правило интегрирования в параметрической области.
\end{itemize}
Каждый из этих трёх алгоритмов определён через интерфейсы
\begin{itemize}
\item \cvar{IElementGeometry}
\item \cvar{IElementBasis}
\item \cvar{IElementIntegrals}
\end{itemize}
Определение конечного элемента заключается в задании конкретных реализаций этих интерфейсов.

\subsubsubsection{Определение линейного одномерного элемента}
\label{sec:linear_segment_assembly}.
Так, в рассматриваемом нами тесте \ename{"[poisson1-fem-linsegm]"}, используются только линейные одномерные элементы.
Используется следующее определение элемента:
\clisting{open}{"test/poisson_fem_solve_test.cpp"}
\clisting{pass}{"TestPoissonLinearSegmentWorker::build"}
\clisting{lines-range}{"geom =", "FemElement"}
Здесь последовательно определяются:
\begin{itemize}
\item
геометрия отрезка \cvar{geom} -- путём задания двух точек в физичекой плоскости \cvar{p0, p1},
\item
линейный одномерный базис \cvar{basis},
\item
правила интегрирования \cvar{integrals} по параметрическому отрезку $x \in [-1, 1]$
с использованием точных формул. Эти формулы зависят только от матрицы Якоби \cvar{jac} (размерности $1\times1$), которая
вычисляется с использованием геометрических свойств элемента
(в данном случае матрица Якоби постоянная, поэтому её можно вычислять в любой точке параметрической плоскости)
\end{itemize}
Этих трёх алгоритмов достаточно для полного определения конечного элемента \cvar{elem}.

\subsubsubsection{Геометрические свойства элемента}
\clisting{open}{"cfd/fem/fem_element.hpp"}
Интерфейс \cvar{IElementGeometry}, заданный в файле \ename{cfd/fem/fem_element.hpp},
определяет геометрические свойства элемента:
\clisting{block}{"class IElementGeometry"}
Для вычиселния элементных матриц главным геометрическим свойством
элемента является функция для вычисления матрицы Якоби (\cvar{jacobi}).
В простейших реализациях этого интерфейса для сиплексных геометрий
матрица Якоби постоянна для любой точки, то есть функция \cvar{jacobi} возвращает
один и тот же ответ вне зависимости от переданного аргумента.

Кроме того, этот интерфейс предоставляет функции преобразования координат из физического простравнства
в параметрическое и обратно: \cvar{to_parametric}, \cvar{to_physical}. А также задает центральную точку
в параметрическом пространстве \cvar{parametric_center}.

\subsubsubsection{Элементный базис}
Интерфейс для определения локального элементного базиса имеет вид
\clisting{open}{"cfd/fem/fem_element.hpp"}
\clisting{block}{"class IElementBasis"}
Этот интерфейс работает только с параметрическим пространстсвом
и определяет следующие методы:
\begin{itemize}
\item \cvar{size} -- количество базисных функций;
\item \cvar{parametric_reference_points} -- вектор из параметрических коордианат
      точек, приписанных к соответствующим базисам;
\item \cvar{value} -- значение базисных функций в заданной точке;
\item \cvar{grad} -- градиент (в параметрическом пространстве) базисных функций по заданным точкам.
\item \cvar{basis_type} -- тип базисной функции. До сих пор мы имели дело только с узловыми (\cvar{BasisType::Nodal})
      функциями.
\item \cvar{upper_hessian} -- верхняя часть матрицы Гессе (вторые производные базисных функций в заданной точке)
\end{itemize}

Конкретная реализация для линейного треугольного элемента \cvar{TriangleLinearBasis}
(в файле \ename{cfd/fem/elem2d/triangle_linear.cpp})
включает в себя линейный Лагранжев базис в двумерном пространстве согласно
\cref{eq:triangle_linear_basis}:
\clisting{open}{"cfd/fem/elem2d/triangle_linear.cpp"}
\clisting{block}{"TriangleLinearBasis::size"}
\clisting{block}{"TriangleLinearBasis::parametric_reference_points"}
\clisting{block}{"TriangleLinearBasis::value"}
\clisting{block}{"TriangleLinearBasis::grad"}


\subsubsubsection{Калькулятор элементных матриц}
Интерфейс \cvar{IElementIntegrals} предоставляет методы
для вычисления элементных матриц.
До сих пор были рассмотрены две элементные матрицы: матрица масс \cref{eq:mass_matrix} и матрица жёсткости \cref{eq:stiff_matrix}.
Для вычисления этих матриц используются функции \cvar{mass_matrix}, \cvar{stiff_matrix}.
Возвращают эти функции локальные квадратные матрицы с числом строк,
равным количеству базисов в элементе. 
Выходные матрицы развёрнуты в линейный массив. Так, для треугольного элемента
с тремя базисными функциями на выходе будет массив из девяти элементов:
$$
m_{00}, m_{01}, m_{02}, m_{10}, m_{11}, m_{12}, m_{20}, m_{21}, m_{22}.
$$

Два подхода к вычислению элементных интегралов: точное и численное интегрирование,
отражены в разных реализациях этого интерфейса.

До сих пор мы рассматривали только точное вычисление.
Точные формулы интегрирования зависят
от вида элемента. Так, для линейного одномерного элемента
аналитическое интегрирование реализовано в классе
\cvar{SegmentLinearIntegrals} в файле \ename{cfd/fem/elem1d/segment_linear.hpp},
а для линейного треугольного элемента -- 
\cvar{TriangleLinearIntegrals} в файле \ename{cfd/fem/elem2d/triangle_linear.hpp}
Все интегралы для этих симлексных элементов будут зависеть только от матрицы Якоби, которая и передётся этому классу в конструктор.
Например, вычисление матрицы масс (по \cref{eq:mass_matrix_lintri}) запрограммировано в виде
\clisting{open}{"cfd/fem/elem2d/triangle_linear.cpp"}
\clisting{block}{"mass_matrix()"}

\subsection{Задание для самостоятельной работы}
\label{sec:fem_programming_problem}
\begin{itemize}
\item
Показать второй порядок аппроксимации решения одномерного уравнения Пуассона
на линейных конечных элементах;
\item
Решить двумерное уравнение Пуассона с граничными условиями первого рода в квадтратной области на треугольной сетке.
Определить порядок аппроксимации двумерного уравнения Пуассона на треугольных элементах.
Для построения треугольных сеток различного разрешения использовать
скрипт \ename{trigrid.py}.
\item
Показать второй порядок аппроксимации решения двумерного уравнения Пуассона
на линейных треугольных конечных элементах;
\item
Сравнить сходимость на сгущающейся сетке конечноэлементного и конечнообъёмного решения
двумерного уравнения Пуассона на одних и тех же треугольных сетках.
Конечнообъёмное решение получать с использованием поправки на скошенность.
\end{itemize}

\paragraph{Рекомендации к программированию}
Для реализации двумерного решения
следует по аналогии с одномерным написать класс
\cvar{ITestPoisson2FemWorker},
в котором реализовать двумерные функции точного решения и правой части,
и наследуемый от него рабочий класс 
\cvar{TestPoissonLinearSegmentWorker},
в котором реализовать статическую функцию \cvar{build_fem}.
При реализации последней
использовать
алгоритмы для треугольников.
\begin{cppcode}
auto geom = std::make_shared<TriangleLinearGeometry>(p0, p1, p2);
auto basis = std::make_shared<TriangleLinearBasis>();
auto integrals = std::make_shared<TriangleLinearIntegrals>(geom->jacobi({}));
\end{cppcode}
